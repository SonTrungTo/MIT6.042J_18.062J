\chapter{The Well Ordering Principle (WOP)}

\begin{center}
    \emph{Every nonempty set of nonnegative integers has a smallest element.}
\end{center}

\begin{claim}
    $\sqrt{2}$ is irrational implies the well ordering principle has been assumed.
\end{claim}

\begin{proof}
    Assume by contradiction that $\sqrt{2} = \dfrac{m}{n}$ where it cannot be rewritten
    in the lowest common denominator. Let $\ms{C}$ be the set of all numerators of such fractions.
    By assumption, $\ms{C}$ is non-empty as $m \in \ms{C}$. By WOP, there exists a smallest element
    $m_{0}$. By definition of $\ms{C}$, there exists $n_{0} > 0, n \in \ms{Z}$ such that
    \begin{equation*}
        \dfrac{m_0}{n_0} \in \ms{C}
    \end{equation*}
    By definition of $\ms{C}$, there must be an integer $k > 1$ such that
    \begin{equation*}
        \dfrac{\dfrac{m_0}{k}}{\dfrac{n_0}{k}} = \dfrac{m_0}{n_0}
    \end{equation*}
    Therefore, $\dfrac{m_0}{k} \in \ms{C}$. But then $\dfrac{m_0}{k} < m_0$, contradicting
    that $m_0$ is the smallest element in $\ms{C}$. Therefore, $\ms{C} = \emptyset$, meaning
    there is no fractions that cannot be written in the lowest common denominator.
\end{proof}

\begin{claim}
    \begin{equation} \label{eq:2.1}
        \sum_{1}^{n} i = n(n+1)/2
    \end{equation}
\end{claim}

\begin{proof}
    The proof is by contradiction. Suppose that there is a set of $\ms{C}$ such that,
    \begin{equation*}
        \ms{C} = \left\{ n \in \ms{N} \ \left| \ \sum_{1}^{n} i \neq n(n + 1)/2 \right. \right\}
    \end{equation*}
    By assumption, $\ms{C}$ is a non-empty set of non-negative integers. By WOP, there is the smallest
    element $n_0 \in \ms{C}$ such that \ref{eq:2.1} is satisfied when $n < n_0$. Since \ref{eq:2.1} is true
    when $n = 0$, it follows that $n_0 > 0$. Therefore, the equation \ref{eq:2.1} must hold true for all
    $0 < n_0 - 1 < n_0$; this means that,
    \begin{equation*}
        1 + 2 + \ldots + n_0 - 1 = \dfrac{(n_0 - 1)n_0}{2}
    \end{equation*}
    Adding $n_0$ to both sides,
    \begin{equation*}
        1 + 2 + \ldots + (n_0 - 1) + n_0 = \dfrac{n_0(n_0 + 1)}{2}
    \end{equation*}
    Therefore, the equation \ref{eq:2.1} holds true for $n_0$, but this contradicts
    that ${n_0 \in \ms{C}}$.
\end{proof}

\note{Unique Factorization Theorem}, also known as \note{Prime Factorization Theorem}
and the \note{Fundamental Theorem of Arithmetic}, states that every integer greater
than one has a unique expression as a product of prime numbers.

\begin{claim}
    Every positive integer greater than one can be factored as a product
    of primes.
\end{claim}

\begin{proof}
    Let $\ms{C}$ be the set of all positive integers greater than one
    that cannot be factored as a product of primes. Suppose by contradiction
    that $\ms{C} \neq \emptyset$. By WOP, there exists a smallest $c_0 \in \ms{C}$ such that $c_0 = c_1 \times c_2$ where
    $c_1$ and $c_2$ are nonnegative and non-prime integers. Since $c_1, c_2 \notin \ms{C}$,
    they are factored as a product of primes,
    \begin{equation*}
        c_1 \times c_2 = p_{11}p_{21}\ldots\ p_{k1}p_{12}p_{22}\ldots\ p_{j2}
    \end{equation*}
    for some postive integers $k, j$. However, this implies that $c_0$ can be factored
    into a product of primes, contradicting that $c_0 \in \ms{C}$.
\end{proof}

\begin{definition}[\boldText{\textit{Well-ordered of a set}}]
    A set of real numbers is \textit{well ordered} when EACH of its NONEMPTY SUBSETS
    HAS a minimum element.
\end{definition}

\begin{ab}
    The set of \note{nonnegative integers} is well ordered. So is $r\ms{N}$

    The set of \note{nonnegative rationals} has minimal element but is \note{not}
    well ordered.
\end{ab}

\begin{ab}
    Well ordering commonly comes up in computer science as a method for proving
    that computations won't run forever.
\end{ab}

\begin{claim}
    For any nonnegative integer $n$ the set of integers greater than or
    equal to $-n$ is well ordered.
\end{claim}

\begin{proof}
    Let $\ms{C}$ be any nonempty set of integers greater than or equal to $-n$.
    Adding $n$ to all elements in $\ms{C}$, the set becomes $\ms{C} + n$. By WOP,
    $\ms{C} + n$ is well ordered. By definition, every nonempty set of $\ms{C} + n$
    has the smallest element $m$. Then every nonempty set of $\ms{C}$ has
    the smallest element $m - n$.
\end{proof}

\begin{definition}[\boldText{Lower bound, Upper bound}]
    A \textit{lower bound} (respectively, \textit{upper bound}) for a set $S$ of real
    numbers is a number $b$ such that $b \leq s$ (respectively, $b \geq s$) for every $s \in S$.
\end{definition}

Well ordered sets' definition lead to two corollaries,

\begin{cor} \label{corollary2.0.1:ch2}
    Any set of integers with a lower bound is well ordered.
\end{cor}

\begin{proof}
    Let $\ms{B}$ be an arbitrary set of integers with a lower bound $b$. Then
    $\ms{B} - b$ has a lower bound $0$. By WOP, $\ms{B} - b$ is well ordered.
    Therefore $\ms{B}$ is well ordered.
\end{proof}

\begin{cor}
    Any nonempty set of integers with an upper bound has a maximal element.
\end{cor}

\begin{proof} \label{corollary2.0.2:ch2}
    Let $\ms{B}$ be an arbitrary nonempty set of integers with an upper
    bound $\lceil b \rceil$. Then $-\ms{B}$ must have a lower bound $-\lceil b \rceil$.
    By the above Corollary \ref{corollary2.0.1:ch2}, $-\ms{B}$ is well ordered and has
    a minimal element. Hence $\ms{B}$ has a maximal element.
\end{proof}

Finite sets are yet another routine example of well ordered sets.

\begin{ab}
    Note that \note{finite} is the basis for WOP.
\end{ab}

\begin{lemNotes}
    Every nonempty finite set of real numbers is well ordered.
\end{lemNotes}

\begin{proof}
    If a nonempty set of real numbers is finite, its number of subsets is also finite.
    Therefore it is sufficient to prove that every such finite set of real numbers has a minimal element.

    We shall prove by contradiction using WOP on the \note{size} of finite sets.

    Suppose we have a set $\ms{C}$ of $n$ positive integers where finite
    sets of size $n$ real numbers have no minimal element. For the sake of our argument,
    $\ms{C} \neq \emptyset$. By WOP, $\ms{C}$ has a minimal element $n_0$. We argue
    that $n_0 > 1$ since a set of real numbers with one single element has
    minimal element as the one element itself.

    Now let $\ms{S}$ be a finite set of real numbers which satisfies
    $\ms{C}$. Then $\ms{S}$ must have at least $n_0$ elements. Let $r_1 \in \ms{S}$.
    By assumption, $\ms{S}$ has no minimal elements. Removing $r_1$ from $\ms{S}$
    means that the set now has a minimal element $r_0$ since the set is finite.
    But then this means that $\min(r_0, r_1)$ is the minimal element of $\ms{S}$,
    contradicting that $\ms{S}$ has no minimal element.
\end{proof}
