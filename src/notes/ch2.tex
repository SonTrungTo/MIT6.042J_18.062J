\chapter{The Well Ordering Principle (WOP)}

\begin{center}
    \emph{Every nonempty set of nonnegative integers has a smallest element.}
\end{center}

\begin{claim}
    $\sqrt{2}$ is irrational implies the well ordering principle has been assumed.
\end{claim}

\begin{proof}
    Assume by contradiction that $\sqrt{2} = \dfrac{m}{n}$ where it cannot be rewritten
    in the lowest common denominator. Let $\ms{C}$ be the set of all numerators of such fractions.
    By assumption, $\ms{C}$ is non-empty as $m \in \ms{C}$. By WOP, there exists a smallest element
    $m_{0}$. By definition of $\ms{C}$, there exists $n_{0} > 0, n \in \ms{Z}$ such that
    \begin{equation*}
        \dfrac{m_0}{n_0} \in \ms{C}
    \end{equation*}
    By definition of $\ms{C}$, there must be an integer $k > 1$ such that
    \begin{equation*}
        \dfrac{\dfrac{m_0}{k}}{\dfrac{n_0}{k}} = \dfrac{m_0}{n_0}
    \end{equation*}
    Therefore, $\dfrac{m_0}{k} \in \ms{C}$. But then $\dfrac{m_0}{k} < m_0$, contradicting
    that $m_0$ is the smallest element in $\ms{C}$. Therefore, $\ms{C} = \emptyset$, meaning
    there is no fractions that cannot be written in the lowest common denominator.
\end{proof}

\begin{claim}
    \begin{equation} \label{eq:2.1}
        \sum_{1}^{n} i = n(n+1)/2
    \end{equation}
\end{claim}

\begin{proof}
    The proof is by contradiction. Suppose that there is a set of $\ms{C}$ such that,
    \begin{equation*}
        \ms{C} = \left\{ n \in \ms{N} \ \left| \ \sum_{1}^{n} i \neq n(n + 1)/2 \right. \right\}
    \end{equation*}
    By assumption, $\ms{C}$ is a non-empty set of non-negative integers. By WOP, there is the smallest
    element $n_0 \in \ms{C}$ such that \ref{eq:2.1} is satisfied when $n < n_0$. Since \ref{eq:2.1} is true
    when $n = 0$, it follows that $n_0 > 0$. Therefore, the equation \ref{eq:2.1} must hold true for all
    $0 < n_0 - 1 < n_0$; this means that,
    \begin{equation*}
        1 + 2 + \ldots + n_0 - 1 = \dfrac{(n_0 - 1)n_0}{2}
    \end{equation*}
    Adding $n_0$ to both sides,
    \begin{equation*}
        1 + 2 + \ldots + (n_0 - 1) + n_0 = \dfrac{n_0(n_0 + 1)}{2}
    \end{equation*}
    Therefore, the equation \ref{eq:2.1} holds true for $n_0$, but this contradicts
    that ${n_0 \in \ms{C}}$.
\end{proof}
