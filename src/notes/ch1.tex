\chapter{What is a Proof?}
\section{Propositions}
\begin{definition}
    A \note{proposition} is a statement (communication) that is either
    true or false.
\end{definition}
\begin{claim}
    $\forall$ $n \in \mathbb{N}$, $p::=n^2+n+41$ is prime
\end{claim}
\boldText{Question: } Is this claim true or false?
\begin{claim}
    No polynomial with integer coefficients can map all nonnegative numbers
    into primes, unless it's a constant.
\end{claim}
\boldText{Question: } Is this true or false?
\begin{claim}[Euler's Conjecture]
    $\forall a,b,c,d \in \mathbb{Z^{+}}$. $a^4 + b^4 + c^4 \neq d^4$
\end{claim}
\begin{claim}
    $313(x^3 + y^3) = z^3$ has no solution when $x,y,z \in \mathbb{Z^{+}}$
\end{claim}
\begin{claim}[Four Color Theorem]
    Every map can be colored with $4$ colors so that adjacent regions have different
    colors.
\end{claim}
\begin{claim}[Fermat's Last Theorem]
    $\forall a,b,c \in \mathbb{Z^{+}}\;\forall n > 2, n \in \mathbb{Z}$. $a^n + b^n \neq c^n$
\end{claim}
\begin{claim}[Goldbach]
    Every even integer greater than $2$ is the sum of two primes.
\end{claim}

\section{Predicates}
\begin{definition}
    A \note{predicate} is a proposition whose truth depends on the value of one
    or more variables.
\end{definition}
If $P$ is a predicate, then $P(n)$ is either \note{true} or \note{false}, depending on
the value of $n$.

\section{The Axiomatic Method}
\begin{definition}
    A \note{proof} is a sequence of logical deductions from a set of axioms
    and previous proved propositions that concludes with the proposition in question.
\end{definition}
\begin{itemize}
    \item \note{Theorems}
    \item \note{Lemma}
    \item \note{Corollary} 
\end{itemize}
$\Rightarrow$ Axiomatic Method

\section{Our axioms}
\subsection{Logical deductions}
Keywords: \note{
    Logical deductions(inference rules), antecedents, conclusion, modus ponens
}
\subsection{Patterns of Proof}
Many proofs follow specific templates\ldots Many special techniques later on.

\section{Proving an Implication}
\begin{definition}
    \note{Implications} means $P \Rightarrow Q$
\end{definition}
\subsection{Method \#1: \texorpdfstring{$P \Rightarrow Q$}{TEXT}}
\subsection{Method \#2: Contrapositive: \texorpdfstring{$\neg Q \Rightarrow \neg P$}{TEXT}}

\section{Proving an ``if and only if''}
\subsection{Method \#1: Prove each statement implies the other}
\subsection{Method \#2: Construct a chain of iffs}

\section{Proof by Cases}
Amusing theorem
\begin{thm}
    Every collection of 6 people includes a club of 3 people or a group of 3 strangers.
\end{thm}
\begin{proof}
    The proof is by case analysis. Let $x$ be one of those $6$ people. Among $5$ other people,
    there are two scenarios:
    \begin{enumerate}
        \item At least $3$ people have met $x$
        \item At least $3$ people have not met $x$
    \end{enumerate}
    We argue that these two cases are exhaustive since we are dividing the 5 people
    into two groups: those who have met $x$ and those who have not.
    \begin{casesp}
        \item \label{Case1:sec1.7:ch1} Suppose that at least $3$ people have met $x$
        
        This is divided further more into two subcases:

        \begin{casesp}
            \item No pairs among those people have met each other. In this case,
            they form a group of at least $3$ strangers. Thus, the theorem holds in this
            subcase.
            \item At least one pair in those people have met. Adding $x$ to such pair
            forms a club of at least $3$ people. The theorem is proved in this subcase.
        \end{casesp}

        This implies that the theorem holds for Case \ref{Case1:sec1.7:ch1}.

        \item \label{Case2:sec1.7:ch1} Suppose that at least $3$ people have not met $x$
        
        This again splits the case into two subcases:

        \begin{casesp}
            \item All pairs among those people have met each other. In this case,
            they form a club of at least $3$ people. Thus the theorem holds in this subcase.
            \item At least one pair in those people have not met. Adding $x$ to
            such pair forms a group of at least $3$ strangers. The theorem holds in this subcase.
        \end{casesp}

        This implies that the theorem holds for Case \ref{Case2:sec1.7:ch1}.

    \end{casesp}
    We have proved the theorem.
\end{proof}

\section{Proof by Contradiction}
\begin{thm}
    $\sqrt{2}$ is irrational
\end{thm}
\begin{proof}
    We use proof by contradiction. Suppose $\sqrt{2}$ is rational, then $\sqrt{2}=\dfrac{p}{q}$,
    where $p$ and $q$ are integers that have no common factors. Then $2=\dfrac{p^2}{q^2}$,
    which means $p^2=2q^2$. Since $p^2$ is even, $p$ must be even (easily proved by contradiction again).
    W.l.o.g, assume $p=2k$ for some integer $k$. Then $4k^2=2q^2 \Rightarrow q^2=2k^2$, which
    implies that $q$ is also even. However, this contradicts the fact that $p$
    and $q$ have no common factors. Therefore $\sqrt{2}$ is irrational.
\end{proof}
