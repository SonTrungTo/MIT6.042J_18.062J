\chapter{The Well Ordering Principle (WOP)}

\begin{pr}
    We shall prove the following statement,
    \begin{lemPr}
        Every amount of postage that can be assembled using only $10$ cent
        and $15$ cent stamps is divisible by $5$.
    \end{lemPr}

    \begin{proof}
        Let the notation "$j\left|\right.k$" indicate the integer $j$
        is a divisor of integer $k$, and let $S(n)$ mean that exactly $n$ cents
        postage can be assembled using only $10$ and $15$ cent stamps. Then
        the proof shows that
        \begin{equation} \label{statement:pr1}
            S(n) \quad \text{IMPLIES} \quad 5 \left|\right. n, \qquad \text{for all nonnegative integers n.}
        \end{equation}
        Let $C$ be the set of \note{counterexamples} to (\ref{statement:pr1}), namely
        \begin{equation*}
            C ::= \left\{ n \left|\right. S(n) \ \text{DOES NOT IMPLY} \ 5 \left|\right. n \ \text{for some} \ n \in S(n) \right\}
        \end{equation*}
        Assume for the purpose of obtaining a contradiction that $C \neq \emptyset$.
        By WOP, there exists a smallest $m \in C$. This $m$ must be positive
        because $5 \left|\right. 0$.
        \beautyBr
        But if $S(m)$ holds and $m$ is positive, then $S(m - 10)$ or $S(m - 15)$
        must hold, because for all positive $m = 10i + 15j$ for either $i > 0$ or $j > 0$; if
        $i > 0$, $(m - 10) \in S(m - 10)$; if $j > 0$, $(m - 15) \in S(m - 15)$.
        \beautyBr
        So suppose $S(m - 10)$ holds. Then $5 \left|\right. m - 10$, because $(m - 10) \notin C$.
        \beautyBr
        But this means that $5 \left|\right. m$, contradicting the fact that
        $m$ is a counterexample.
        \beautyBr
        Next, if $S(m - 15)$ holds, we arrive at a contradiction the same way.
        \beautyBr
        Since we get a contradiction in both cases, we conclude that $C = \emptyset$,
        which proves that $(\ref{statement:pr1})$ holds.
    \end{proof}
\end{pr}

\begin{pr}
    Note that $1 \in C$. This means that $0 < m \leq 1$. Since $m \in \ms{N}$,
    $m = 1$. However, $F(m - 2) = F(-1)$ does not exist: the proof is bogus.
\end{pr}

\begin{pr}
    WOP cannot be applied on the set of rational numbers. Indeed, the set
    of positive rational numbers have no minimum element.
\end{pr}
